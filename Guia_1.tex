\documentclass{article}\usepackage[]{graphicx}\usepackage[]{color}
%% maxwidth is the original width if it is less than linewidth
%% otherwise use linewidth (to make sure the graphics do not exceed the margin)
\makeatletter
\def\maxwidth{ %
  \ifdim\Gin@nat@width>\linewidth
    \linewidth
  \else
    \Gin@nat@width
  \fi
}
\makeatother

\definecolor{fgcolor}{rgb}{0.345, 0.345, 0.345}
\newcommand{\hlnum}[1]{\textcolor[rgb]{0.686,0.059,0.569}{#1}}%
\newcommand{\hlstr}[1]{\textcolor[rgb]{0.192,0.494,0.8}{#1}}%
\newcommand{\hlcom}[1]{\textcolor[rgb]{0.678,0.584,0.686}{\textit{#1}}}%
\newcommand{\hlopt}[1]{\textcolor[rgb]{0,0,0}{#1}}%
\newcommand{\hlstd}[1]{\textcolor[rgb]{0.345,0.345,0.345}{#1}}%
\newcommand{\hlkwa}[1]{\textcolor[rgb]{0.161,0.373,0.58}{\textbf{#1}}}%
\newcommand{\hlkwb}[1]{\textcolor[rgb]{0.69,0.353,0.396}{#1}}%
\newcommand{\hlkwc}[1]{\textcolor[rgb]{0.333,0.667,0.333}{#1}}%
\newcommand{\hlkwd}[1]{\textcolor[rgb]{0.737,0.353,0.396}{\textbf{#1}}}%

\usepackage{framed}
\makeatletter
\newenvironment{kframe}{%
 \def\at@end@of@kframe{}%
 \ifinner\ifhmode%
  \def\at@end@of@kframe{\end{minipage}}%
  \begin{minipage}{\columnwidth}%
 \fi\fi%
 \def\FrameCommand##1{\hskip\@totalleftmargin \hskip-\fboxsep
 \colorbox{shadecolor}{##1}\hskip-\fboxsep
     % There is no \\@totalrightmargin, so:
     \hskip-\linewidth \hskip-\@totalleftmargin \hskip\columnwidth}%
 \MakeFramed {\advance\hsize-\width
   \@totalleftmargin\z@ \linewidth\hsize
   \@setminipage}}%
 {\par\unskip\endMakeFramed%
 \at@end@of@kframe}
\makeatother

\definecolor{shadecolor}{rgb}{.97, .97, .97}
\definecolor{messagecolor}{rgb}{0, 0, 0}
\definecolor{warningcolor}{rgb}{1, 0, 1}
\definecolor{errorcolor}{rgb}{1, 0, 0}
\newenvironment{knitrout}{}{} % an empty environment to be redefined in TeX

\usepackage{alltt}
\IfFileExists{upquote.sty}{\usepackage{upquote}}{}
\begin{document}
\begin{knitrout}
\definecolor{shadecolor}{rgb}{0.969, 0.969, 0.969}\color{fgcolor}\begin{kframe}
\begin{alltt}
\hlcom{#GUIA 1}

\hlcom{#Calculos numericos}
\hlnum{2}\hlopt{*}\hlstd{(}\hlnum{3}\hlopt{+}\hlnum{4}\hlstd{)}\hlopt{^}\hlnum{2}
\end{alltt}
\begin{verbatim}
## [1] 98
\end{verbatim}
\begin{alltt}
\hlkwd{sqrt}\hlstd{(}\hlnum{16}\hlstd{)}
\end{alltt}
\begin{verbatim}
## [1] 4
\end{verbatim}
\begin{alltt}
\hlkwd{abs}\hlstd{(}\hlopt{-}\hlnum{97.6}\hlstd{)} \hlcom{# abs(x) calcula el valor absoluto de x2*(3+4)^2}
\end{alltt}
\begin{verbatim}
## [1] 97.6
\end{verbatim}
\begin{alltt}
\hlstd{x} \hlkwb{=} \hlnum{4} \hlcom{# almacena el valor de 4 en la variable x}
\hlstd{x} \hlcom{# Muestra el contenido de la variable x}
\end{alltt}
\begin{verbatim}
## [1] 4
\end{verbatim}
\begin{alltt}
\hlkwd{sqrt}\hlstd{(x)}\hlopt{-}\hlnum{3}\hlopt{/}\hlnum{2}
\end{alltt}
\begin{verbatim}
## [1] 0.5
\end{verbatim}
\begin{alltt}
\hlstd{p} \hlkwb{<-} \hlstd{(}\hlnum{4} \hlopt{>} \hlnum{8}\hlstd{)}
\hlstd{p}
\end{alltt}
\begin{verbatim}
## [1] FALSE
\end{verbatim}
\begin{alltt}
\hlstd{q} \hlkwb{=} \hlopt{-}\hlnum{6}\hlopt{+}\hlnum{4} \hlopt{<} \hlnum{3} \hlopt{&&} \hlnum{4} \hlopt{!=} \hlnum{10}
\hlstd{q}
\end{alltt}
\begin{verbatim}
## [1] TRUE
\end{verbatim}
\begin{alltt}
\hlstd{r} \hlkwb{=} \hlopt{-}\hlnum{6}\hlopt{+}\hlnum{4} \hlopt{>} \hlnum{3} \hlopt{||} \hlnum{4} \hlopt{==} \hlnum{10}
\hlstd{r}
\end{alltt}
\begin{verbatim}
## [1] FALSE
\end{verbatim}
\begin{alltt}
\hlstd{t} \hlkwb{<-} \hlopt{!}\hlstd{r}
\hlstd{t}
\end{alltt}
\begin{verbatim}
## [1] TRUE
\end{verbatim}
\begin{alltt}
\hlkwd{sin}\hlstd{(pi}\hlopt{/}\hlnum{2}\hlstd{)}
\end{alltt}
\begin{verbatim}
## [1] 1
\end{verbatim}
\begin{alltt}
\hlstd{(y}\hlkwb{=}\hlkwd{cos}\hlstd{(pi))} \hlcom{# Los primeros par�ntesis permiten ver el valor calculado de y}
\end{alltt}
\begin{verbatim}
## [1] -1
\end{verbatim}
\begin{alltt}
\hlkwd{log}\hlstd{(}\hlnum{3}\hlstd{)} \hlcom{# Calcula el logaritmo natural de 3}
\end{alltt}
\begin{verbatim}
## [1] 1.098612
\end{verbatim}
\begin{alltt}
\hlkwd{log10}\hlstd{(}\hlnum{8}\hlstd{)} \hlcom{# Calcula el logaritmo base 10 de 8}
\end{alltt}
\begin{verbatim}
## [1] 0.90309
\end{verbatim}
\begin{alltt}
\hlcom{# La sintaxis general es: logb(x, base)}
\hlkwd{logb}\hlstd{(}\hlnum{16}\hlstd{,} \hlnum{7}\hlstd{)}
\end{alltt}
\begin{verbatim}
## [1] 1.424829
\end{verbatim}
\begin{alltt}
\hlcom{# exp() c�lcula la funci�n exponencial}
\hlkwd{exp}\hlstd{(}\hlnum{1}\hlstd{)}
\end{alltt}
\begin{verbatim}
## [1] 2.718282
\end{verbatim}
\end{kframe}
\end{knitrout}

\end{document}
