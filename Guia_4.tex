\documentclass{article}\usepackage[]{graphicx}\usepackage[]{color}
%% maxwidth is the original width if it is less than linewidth
%% otherwise use linewidth (to make sure the graphics do not exceed the margin)
\makeatletter
\def\maxwidth{ %
  \ifdim\Gin@nat@width>\linewidth
    \linewidth
  \else
    \Gin@nat@width
  \fi
}
\makeatother

\definecolor{fgcolor}{rgb}{0.345, 0.345, 0.345}
\newcommand{\hlnum}[1]{\textcolor[rgb]{0.686,0.059,0.569}{#1}}%
\newcommand{\hlstr}[1]{\textcolor[rgb]{0.192,0.494,0.8}{#1}}%
\newcommand{\hlcom}[1]{\textcolor[rgb]{0.678,0.584,0.686}{\textit{#1}}}%
\newcommand{\hlopt}[1]{\textcolor[rgb]{0,0,0}{#1}}%
\newcommand{\hlstd}[1]{\textcolor[rgb]{0.345,0.345,0.345}{#1}}%
\newcommand{\hlkwa}[1]{\textcolor[rgb]{0.161,0.373,0.58}{\textbf{#1}}}%
\newcommand{\hlkwb}[1]{\textcolor[rgb]{0.69,0.353,0.396}{#1}}%
\newcommand{\hlkwc}[1]{\textcolor[rgb]{0.333,0.667,0.333}{#1}}%
\newcommand{\hlkwd}[1]{\textcolor[rgb]{0.737,0.353,0.396}{\textbf{#1}}}%

\usepackage{framed}
\makeatletter
\newenvironment{kframe}{%
 \def\at@end@of@kframe{}%
 \ifinner\ifhmode%
  \def\at@end@of@kframe{\end{minipage}}%
  \begin{minipage}{\columnwidth}%
 \fi\fi%
 \def\FrameCommand##1{\hskip\@totalleftmargin \hskip-\fboxsep
 \colorbox{shadecolor}{##1}\hskip-\fboxsep
     % There is no \\@totalrightmargin, so:
     \hskip-\linewidth \hskip-\@totalleftmargin \hskip\columnwidth}%
 \MakeFramed {\advance\hsize-\width
   \@totalleftmargin\z@ \linewidth\hsize
   \@setminipage}}%
 {\par\unskip\endMakeFramed%
 \at@end@of@kframe}
\makeatother

\definecolor{shadecolor}{rgb}{.97, .97, .97}
\definecolor{messagecolor}{rgb}{0, 0, 0}
\definecolor{warningcolor}{rgb}{1, 0, 1}
\definecolor{errorcolor}{rgb}{1, 0, 0}
\newenvironment{knitrout}{}{} % an empty environment to be redefined in TeX

\usepackage{alltt}
\IfFileExists{upquote.sty}{\usepackage{upquote}}{}
\begin{document}

\begin{knitrout}
\definecolor{shadecolor}{rgb}{0.969, 0.969, 0.969}\color{fgcolor}\begin{kframe}
\begin{alltt}
\hlcom{#GUIA 4}

\hlstd{Entrada1} \hlkwb{<-} \hlkwd{read.table}\hlstd{(}\hlstr{"datos01.txt"}\hlstd{,} \hlkwc{header}\hlstd{=}\hlnum{TRUE}\hlstd{)}
\end{alltt}


{\ttfamily\noindent\color{warningcolor}{\#\# Warning in read.table("{}datos01.txt"{}, header = TRUE): incomplete final line found by readTableHeader on 'datos01.txt'}}\begin{alltt}
\hlstd{Entrada1}
\end{alltt}
\begin{verbatim}
##   Edad Estatura Peso sexo
## 1   26     1.65  146    F
## 2   21     1.73  158    M
## 3   21     1.81  167    M
## 4   20     1.70  152    F
\end{verbatim}
\begin{alltt}
\hlstd{Edat1} \hlkwb{<-} \hlkwd{scan}\hlstd{(}\hlstr{"datos01.txt"}\hlstd{,} \hlkwd{list}\hlstd{(}\hlkwc{X1}\hlstd{=}\hlnum{0}\hlstd{,} \hlkwc{X2}\hlstd{=}\hlnum{0}\hlstd{),} \hlkwc{skip} \hlstd{=} \hlnum{1}\hlstd{,} \hlkwc{flush} \hlstd{=} \hlnum{TRUE}\hlstd{,} \hlkwc{quiet} \hlstd{=} \hlnum{TRUE}\hlstd{)}
\hlstd{Edat1}
\end{alltt}
\begin{verbatim}
## $X1
## [1] 26 21 21 20
## 
## $X2
## [1] 1.65 1.73 1.81 1.70
\end{verbatim}
\begin{alltt}
\hlstd{pp} \hlkwb{<-} \hlkwd{scan}\hlstd{(}\hlstr{"datos02.txt"}\hlstd{,} \hlkwc{skip} \hlstd{=} \hlnum{1}\hlstd{,} \hlkwc{quiet}\hlstd{=} \hlnum{TRUE}\hlstd{)}
\hlstd{pp}
\end{alltt}
\begin{verbatim}
## [1]  2  3  5  7 11 13 17
\end{verbatim}
\begin{alltt}
\hlkwd{library}\hlstd{(foreign)}
\end{alltt}


{\ttfamily\noindent\color{warningcolor}{\#\# Warning: package 'foreign' was built under R version 3.2.2}}\begin{alltt}
\hlstd{baseproductos} \hlkwb{<-}\hlkwd{read.table}\hlstd{(}\hlstr{"productos.csv"}\hlstd{,}\hlkwc{header}\hlstd{=}\hlnum{TRUE}\hlstd{,}\hlkwc{sep} \hlstd{=} \hlstr{","}\hlstd{)}
\end{alltt}


{\ttfamily\noindent\color{warningcolor}{\#\# Warning in file(file, "{}rt"{}): no fue posible abrir el archivo 'productos.csv': No such file or directory}}

{\ttfamily\noindent\bfseries\color{errorcolor}{\#\# Error in file(file, "{}rt"{}): no se puede abrir la conexi�n}}\begin{alltt}
\hlstd{baseproductos}
\end{alltt}


{\ttfamily\noindent\bfseries\color{errorcolor}{\#\# Error in eval(expr, envir, enclos): objeto 'baseproductos' no encontrado}}\begin{alltt}
\hlkwd{library}\hlstd{(Hmisc)}
\end{alltt}


{\ttfamily\noindent\color{warningcolor}{\#\# Warning: package 'Hmisc' was built under R version 3.2.2}}

{\ttfamily\noindent\itshape\color{messagecolor}{\#\# Loading required package: grid\\\#\# Loading required package: lattice\\\#\# Loading required package: survival\\\#\# Loading required package: Formula}}

{\ttfamily\noindent\color{warningcolor}{\#\# Warning: package 'Formula' was built under R version 3.2.2}}

{\ttfamily\noindent\itshape\color{messagecolor}{\#\# Loading required package: ggplot2}}

{\ttfamily\noindent\color{warningcolor}{\#\# Warning: package 'ggplot2' was built under R version 3.2.2}}

{\ttfamily\noindent\itshape\color{messagecolor}{\#\# \\\#\# Attaching package: 'Hmisc'\\\#\# \\\#\# The following objects are masked from 'package:base':\\\#\# \\\#\#\ \ \ \  format.pval, round.POSIXt, trunc.POSIXt, units}}\begin{alltt}
\hlstd{Baseimportante}\hlkwb{<-}\hlkwd{spss.get}\hlstd{(}\hlstr{"Mundo.sav"}\hlstd{,}\hlkwc{use.value.labels} \hlstd{=}\hlnum{TRUE}\hlstd{)}
\end{alltt}


{\ttfamily\noindent\bfseries\color{errorcolor}{\#\# Error in read.spss(file, use.value.labels = use.value.labels, to.data.frame = to.data.frame, : unable to open file: 'No such file or directory'}}\begin{alltt}
\hlstd{Baseimportante}
\end{alltt}


{\ttfamily\noindent\bfseries\color{errorcolor}{\#\# Error in eval(expr, envir, enclos): objeto 'Baseimportante' no encontrado}}\end{kframe}
\end{knitrout}
@



\end{document}
